%%%%%%%%%%%%%%%%%%%%%%%%%%%%%%%%%%%%%%%%%%%%%%%%%%%%%%%%%%%%%%%%%%%%%%%%%%%%%%%
% NAME:             booting.tex
%
% AUTHOR:           Ethan D. Twardy <edtwardy@mtu.edu>
%
% DESCRIPTION:      
%
% CREATED:          04/30/2019
%
% LAST EDITED:      05/01/2019
%%%

\documentclass{article}
\usepackage[margin=1in]{geometry}
\usepackage{setspace}
\usepackage{fancyhdr}

\pagestyle{empty}
\pagestyle{fancy}
\fancyhf{}
\lhead{Ground Up Booting}
\setlength{\headheight}{15pt}

\begin{document}
\section{Introduction}
When I first began learning operating systems concepts, I really struggled with
the vast amount of hand-waving that teachers, instructors, and colleagues
tolerated. It's easy to give up on explaining difficult concepts by making an
excuse about the real-world difficulties and minutia that are outside the scope
of the discussion. This guide intends to be a supplemental document to explain,
in \textit{excruciating} depth, the things that get waved away in the
classroom.

\section{System Reset}
% TODO: System Reset

\section{Disks and Partitioning}
This is intended to be a comprehensive guide on the current state of
partitioning methods and technologies. First, a brief history of disk
partitioning will be discussed, along with a list of partition types, and an
explanation of each.

\subsection{Introduction}


\subsection{Master Boot Record (MBR)}

\section{Linux Image Types}
% TODO: Linux Image Types

\end{document}

%%%%%%%%%%%%%%%%%%%%%%%%%%%%%%%%%%%%%%%%%%%%%%%%%%%%%%%%%%%%%%%%%%%%%%%%%%%%%%
