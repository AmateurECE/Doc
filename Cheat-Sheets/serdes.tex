%%%%%%%%%%%%%%%%%%%%%%%%%%%%%%%%%%%%%%%%%%%%%%%%%%%%%%%%%%%%%%%%%%%%%%%%%%%%%%%
% NAME:             serdes.tex
%
% AUTHOR:           Ethan D. Twardy
%
% DESCRIPTION:      This is a Cheat Sheet about Serial transmission methods,
%                   including PCI, SATA, Ethernet, and lower profile
%                   transmission protocols, such as SPI and I2C.
%
% CREATED:          04/18/2018
%
% LAST EDITED:      04/18/2018
%%%

\documentclass[12pt]{article}
\pagestyle{empty} % Prevent style conflicts with `fancy'
\usepackage[margin=1in]{geometry} % Pretty much just to set the margins
\usepackage{fancyhdr} % Header & Footer
\usepackage{setspace} % Spacing.

% Header
% pdflatex will probably complain about \headheight.
\setlength{\headheight}{28pt}
\pagestyle{fancy}
\fancyhf{}
% Header
\rhead{Ethan D.Twardy\\April 18, 2018{}}
\lhead{SerDes Protocols\\Cheat-Sheets}

% This may come in handy.
%% \setlength{\emergencystretch}{15pt}
\setlength{\parskip}{10pt}
\setlength{\parindent}{0pt}

\begin{document}

\section{Ethernet and XAUI}
% OSI Model and Network Device Architecture.
\subsection{Physical Coding Sublayer}
The Physical Coding Sublayer (PCS) is an abstraction layer which provides an
interface between the PMA (Physical Medium Attachment) sublayer and the Media
Independent Interface (MII). It is present in systems which offer greater than
100 Mb/s bandwidth. This functional block is 

\section{PCI Express}
\section{SATA}

\end{document}

%%%%%%%%%%%%%%%%%%%%%%%%%%%%%%%%%%%%%%%%%%%%%%%%%%%%%%%%%%%%%%%%%%%%%%%%%%%%%%%
