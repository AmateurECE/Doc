%%%%%%%%%%%%%%%%%%%%%%%%%%%%%%%%%%%%%%%%%%%%%%%%%%%%%%%%%%%%%%%%%%%%%%%%%%%%%%%
% NAME:             filters.tex
%
% AUTHOR:           Ethan D. Twardy
%
% DESCRIPTION:      A cheat sheet of methods for designing and analyzing first
%                   and second order passive and active filters.
%
% CREATED:          02/20/2018
%
% LAST EDITED:      02/20/2018
%%%

\documentclass[12pt]{article}
\pagestyle{empty} % Prevent style conflicts with `fancy'
\usepackage[margin=1in]{geometry} % Pretty much just to set the margins
\usepackage{fancyhdr} % Header & Footer
\usepackage{setspace} % Spacing.

% Header
% pdflatex will probably complain about \headheight.
%% \setlength{\headheight}{28pt}
\pagestyle{fancy}
\fancyhf{}
% Header
%% \rhead{Ethan D.Twardy}
%% \lhead{\LaTeX Template}

% This may come in handy.
%% \setlength{\emergencystretch}{15pt}
\setlength{\parskip}{10pt}
\setlength{\parindent}{0pt}

\title{This is a template. A \LaTeX template}
\author{Ethan D. Twardy}

\begin{document}
% TODO: Add the method for including a pdf
% TODO: Add the method for including an image
\doublespacing

\maketitle
\pagebreak

% An example of using a table to do a signature line...or something.
Don't forget that the line break command has an optional argument: \\[5pt]
\begin{tabular}{@{}p{2in}p{3in}@{}}
  Signature: & \hrulefill
\end{tabular}

\end{document}

%%%%%%%%%%%%%%%%%%%%%%%%%%%%%%%%%%%%%%%%%%%%%%%%%%%%%%%%%%%%%%%%%%%%%%%%%%%%%%%
