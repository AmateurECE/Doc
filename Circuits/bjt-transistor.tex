%%%%%%%%%%%%%%%%%%%%%%%%%%%%%%%%%%%%%%%%%%%%%%%%%%%%%%%%%%%%%%%%%%%%%%%%%%%%%%%
% NAME:             bjt-transistor.tex
%
% AUTHOR:           Ethan D. Twardy
%
% DESCRIPTION:      A paper about designing REAL circuits with BJTs.
%
% CREATED:          01/03/2018
%
% LAST EDITED:      01/03/2018
%%%

\documentclass[12pt]{article}
\pagestyle{empty} % Prevent style conflicts with `fancy'
\usepackage[margin=1in]{geometry} % Pretty much just to set the margins
\usepackage{fancyhdr} % Header & Footer
\usepackage{setspace} % Spacing.

% Header
% pdflatex will probably complain about \headheight.
%% \setlength{\headheight}{28pt}
\pagestyle{fancy}
\fancyhf{}
% Header
\rhead{Ethan D.Twardy}
\lhead{BJT: Real vs. Ideal}

% This may come in handy.
%% \setlength{\emergencystretch}{15pt}
\setlength{\parskip}{10pt}
\setlength{\parindent}{0pt}

\begin{document}
\doublespacing
%
\begin{abstract}
  When I first began studying active circuits, I noticed that the majority of
  the material available---in textbooks, on the internet, etc---caters
  either to academics or to technicians. In this paper, I will provide a
  detailed comparison of the two schools of analysis. We will examine how to
  analyze bipolar junction transistor circuits using the numerically equivalent
  methods one would develop in an undergraduate course on electronics, and then
  take a step back and use the values from a datasheet to design a functional
  circuit.
\end{abstract}
%
\end{document}

%%%%%%%%%%%%%%%%%%%%%%%%%%%%%%%%%%%%%%%%%%%%%%%%%%%%%%%%%%%%%%%%%%%%%%%%%%%%%%%
