%%%%%%%%%%%%%%%%%%%%%%%%%%%%%%%%%%%%%%%%%%%%%%%%%%%%%%%%%%%%%%%%%%%%%%%%%%%%%%%
% NAME:             SoftwareDevelopmentPlan.tex
%
% AUTHOR:           Ethan D. Twardy <edtwardy@mtu.edu>
%
% DESCRIPTION:      This is my generic software development plan. Used on all
%                   of my projects unless specified otherwise.
%
% CREATED:          06/15/2019
%
% LAST EDITED:      06/15/2019
%%%

\documentclass{designdoc}
\lhead{Software Development Plan}

\begin{document}
\section{Introduction}
This document presents a Software Development Plan for general use. This plan
will be used on all of my projects, unless specified otherwise in the project's
repository \texttt{README.md} file. In some cases, it may be necessary to skip
certain steps of this design.

This design process is meant to be iterative. At each step of the design, the
engineer should consider the implications of the current design on further
levels. It may be possible to repeat certain steps.

The other issue of concern is the FMEA (Failure Mode Effects and Analysis).
This design step should be completed concurrently with the other steps, as the
mitigations put into place to prevent failure should be well integrated into
the design. Quality Engineers would say that the FMEA should ``drive the
design.''

\section{UI Requirements}
In this section, all use cases, interfaces, state machines and other
necessary resources that illustrate direct interactions with the end user are
specified.

\section{System Verification Plan}
This step is used to develop the test plan for verification of the system from
a black-box perspective.

\section{UI Design}
In this step, the UI is actually designed. That is, use case diagrams are used
to describe system state and user interaction.

\section{System Requirements}
These requirements detail high-level features of the system, ideally being
implementation agnostic.

\section{System Design}
The design of the high level components.

\section{Component Interface Verification Test Plan}
A plan is created to test every facet of the interface between high-level
components.

\section{Component Interface Design}
In this section, every facet of the interface between high-level software
components is designed.

\section{Component Design}
Breaks down a single high-level component into a series of atomic Software
Units.

\section{Unit Test Plan}
A test plan is created to verify the interfaces and behaviors of all atomic
Software Units in the system.

\section{Unit Interface Design}
Units and their interfaces are laid out. This design should account for all
Unit Tests.
\end{document}

%%%%%%%%%%%%%%%%%%%%%%%%%%%%%%%%%%%%%%%%%%%%%%%%%%%%%%%%%%%%%%%%%%%%%%%%%%%%%%%
