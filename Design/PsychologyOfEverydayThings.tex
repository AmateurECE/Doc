%%%%%%%%%%%%%%%%%%%%%%%%%%%%%%%%%%%%%%%%%%%%%%%%%%%%%%%%%%%%%%%%%%%%%%%%%%%%%%%
% NAME:             PsychologyOfEverydayThings.tex
%
% AUTHOR:           Ethan D. Twardy <edtwardy@mtu.edu>
%
% DESCRIPTION:      
%
% CREATED:          01/11/2020
%
% LAST EDITED:      01/16/2020
%%%

\documentclass{designdoc}

\lhead{Notes from the Psychology of Everyday Things}

\begin{document}
\section*{Notes from the Psychology of Everyday Things}
\begin{enumerate}
\item Something that happens right after an action appears to be caused by that
  action.
\item Users of devices, upon initial inspection of a device, already begin
  forming a conceptual model of the device and mentally simulate its operation.
  A good conceptual model allows us to accurately predict the effects of our
  actions before we take them. Conceptual models are part of a larger concept
  in design (\textit{mental models}). People form mental models through
  experience. The visible part of the device is called the system image. When
  the system image creates a conceptual model that cannot be used to accurately
  predict the outcomes of all actions that can be taken with the device, this
  is a poor design.
\item There are three kinds of clues that come from the visual structure of a
  device: \textit{affordances}, \textit{constraints}, and \textit{mappings}.
  \begin{enumerate}
  \item \textit{Affordances}: Perceived an actual properties of a thing (e.g. a
    chair \textit{affords} sitting.).
  \item \textit{Constraints}: Limit how a thing can be used (e.g. scissor holes
    prevent too many fingers from being inserted into them, notches and shapes
    prevent parts from being inserted correctly, buttons cannot be turned,
    etc.)
  \item \textit{Mappings}: The relationship between two things (e.g. between
    controls and their movements and the results in the world).
  \end{enumerate}
\item Whenever the number of possible actions exceeds the number of controls,
  there is apt to be difficulty (with few exceptions).
\item \textit{Natural Mappings}: mappings that appear natural to the user (e.g.
  the seat adjustment controls in a Mercedes-Benz, which are shaped like the
  seat and move in analogous ways to the seat).
\item Feedback is very important because it allows users to instantly verify
  that something (ideally, the correct thing) happened when they made an
  action.
\item The Technological paradox: The same technology that simplifies life by
  providing more functions in each device also complicates life by making the
  device harder to learn, harder to use.
\item Added complexity and difficulty cannot be avoided when functions are
  added, but with clever design, they can be minimized.
\item Designers should take special pains to make errors as cost-free as
  possible.
\item If an error is possible, someone will make it. The designer must assume
  that all possible errors will occur and design so as to minimize the chance
  of the error in the first place, or its effects once it gets made. Errors
  should be easy to detect, they should have minimized consequences and, if
  possible, their effects should be reversible.
\item In the absence of feedback or external information, people are free to
  let their imaginations run free as long as the mental models they develop
  account for the facts as they perceive them (consider the false ``folk''
  theory of the thermostat, in which some people believe that turning it up
  higher causes the temperature to rise faster.)
\item Learned Helplessness: People do something wrong one or a few times. As a
  result, they believe that the task cannot be completed, at least not by them.
\item Taught Helplessness: When poorly designed things create equally poor
  mental models that are passed one generationally, and/or when failure at one
  simple task creates failure at other simple tasks which require understanding
  of the first as a prerequisite (e.g. math phobias, where difficulty in one
  subject translate to difficulty in all subsequent subjects).
\item Norman presents this great and complicated model for decision making, in
  which high-level goals are broken down into discrete \textit{intentions} that
  can be completed in sequence to achieve a vaguely-described goal. In general,
  completion of an intention is performed in two stages: \textit{Execution} and
  \textit{Evaluation}. For a device, controls and actions provided by a device
  or system must match, with a reasonably low impedance, the intentions created
  by the user. This interface is the \textit{Gulf of Execution}. It's not only
  important that the device presents actions similar to the intentions, but
  also that the device communicates which intentions are necessary to complete
  a high-level goal.
\item The other gulf is the \textit{Gulf of Evaluation}. This is the amount of
  effort necessary for the user to interpret the actual state of the device to
  determine if the expectations and intentions have been met.
\item Not all the information required for precise behavior has to be in the
  head, for these reasons:
  \begin{enumerate}
  \item Information is in the world: Precise behavior is determined by
    combining the information in the world with that in the head.
  \item Great precision is not required: Perfect precision is seldom required.
    Often information only needs to define sufficient behavior to distinguish
    one action from another.
  \item Natural constraints are present: physical constraints that limit
    possible behavior are present.
  \item Cultural constraints are present: Conventions are present that usually
    suggest correct behavior when it is not provided naturally.
  \end{enumerate}
\item If a design depends on labels, it may be faulty. Labels are important and
  often necessary, but the appropriate use of natural mappings can minimize the
  need for them. Wherever labels seem necessary, consider another design.
\item The problem with switches:
  \begin{enumerate}
  \item The Grouping Problem: Determining which switch goes with which
    function.
  \item The Mapping Problem: When there are many individually-switched lights,
    how to determine which switch controls which light.
  \end{enumerate}
\item Things that can cause trouble should not be placed where they can be
  operated accidentally.
\item Solutions to the grouping problem: group functionally related switches
  together, and away from functionally unrelated switches. Or use different
  switches (e.g. shape coding--wheel-shaped switch for landing gear,
  flap-shaped switch for flaps.)
\item Sound is important as a feedback device. Sound can communicate receipt of
  input, state, etc. It can also be intrusive, annoying, overstimulating,
  unhelpful, and it could also give away information to eavesdroppers.
\item The best way to handle all slips is to eliminate irreversible actions.
\item Designing for Error
  \begin{enumerate}
  \item Understand the causes of error and design to minimize those causes.
  \item Make it possible to reverse actions--to ``undo'' them--or make it
    harder to do what cannot be reversed.
  \item Make it easier to discover the errors that do occur, and make them
    easier to correct.
    \item Change the attitude toward errors. Think of an object's user as
      attempting to do a task, getting there by imperfect approximations. Don't
      think of the user as making errors; think of the actions as
      approximations of what is desired.
  \end{enumerate}
\item Forcing functions: A form of physical constraint--situations in which the
  actions are constrained so that failure at one stage prevents the next step
  from happening. If you are going to use a forcing function, make sure it
  works right, is reliable, and distinguishes legitimate violations from
  illegitimate ones.
\item The designer shouldn't think of a simple dichotomy between errors and
  correct behavior; rather, the entire interaction should be treated as a
  cooperative endeavour between person and machine, one in which misconceptions
  can arise on either side. Assume that every possible mishap will happen, so
  protect against it. Make actions reversible. Try to make them less costly.
  \begin{enumerate}
  \item Put the required knowledge in the world. Don't require all the
    knowledge to be in the head. Do allow for more efficient operation when the
    user has learned the operations.
  \item Use the power of natural and artificial constraints: physical, logical,
    semantic and cultural. Use forcing functions and natural mappings.
  \item Narrow the gulfs of execution and evaluation. Make things visible, both
    for execution and evaluation. On the execution side, make the options
    readily available. On the evaluation side, make the results of each action
    apparent. Make it possible to determine the system state readily, easily,
    and accurately, and in a form consistent with the user's goals, intentions,
    and expectations.
  \end{enumerate}
\item Evolutionary Design: Incremental change between models of a product that
  bring it closer to a better, more usable design. Forces that work against
  evolutionary design:
  \begin{enumerate}
  \item Time: New models are often in design before the previous model has been
    released to customers. There is often also not a reasonable method for
    obtaining customer feedback.
  \item The curse of individuality: There's pressure for each model to be
    unique from the next, or each company to produce a unique model from its
    competitors.
  \end{enumerate}
\item The severe constraints of existing practice prevent change, even where
  the change would be an improvement.
\item Chord keyboards (used by court stenographers) are very hard to learn and
  even harder to retain, because all the knowledge must be in the head.
\item Designers are experts in the product. They are capable of operating a
  device using only knowledge in the head, because they have learned it
  painstakingly and intimately, in a way that any user wil not. They live in a
  world where the gulf of evaluation is infinite. They are separated from the
  end users by multiple layers of bureacracy--leaders that believe they know
  what customers want. If the designer accepts the product specification
  without personal investigation, they are guaranteed to design an inferior
  product.
\item Physical Anthropometry is the name of the field related to determining
  average metrics of the human body.
\item Designers should design for the special cases--the 1\% of users that
  don't fit into the 99\% percentile.
\item More challenges faced by the designer:
  \begin{enumerate}
  \item Creeping Featurism: Users and Consumers continuously ask for more and
    better features. Complexity increases non-linearly with respect to the
    number of features, so adding features dramatically reduces usability over
    time. There are two solutions. First is to simply do without the added
    features. Second is to modularize--group related features together and hide
    their controls away into small and simple functional modules.
  \item Worshipping of False Images: Some consumers seem to worship the
    complexity of some devices, despite their unusability.
  \end{enumerate}
\item Computer systems pose a particularly strong problem for the designer,
  because the user-interfaces are usually at the mercy of the programmer, a
  professional who has no business designing user interfaces.
\item If you want to make complicated devices easier to learn by the user, make
  them explorable. To make a device explorable, the device must:
  \begin{enumerate}
  \item Make all available options visible to the user at all stages of use.
  \item Make all actions and their effects obvious and easily interpretable.
  \item Make all actions reversible.
  \end{enumerate}
\item Basically, how to design for the user:
  \begin{enumerate}
  \item Use both knowledge in the world and knowledge in the head.
  \item Simplify the structure of tasks.
  \item Make things visible; bridge the gulfs of execution and evaluation.
  \item Get the mappings right.
  \item Exploit the power of constraints, both natural and artificial.
  \item Design for error.
  \item When all else fails, standardize.
  \end{enumerate}
\item Three conceptual models:
  \begin{enumerate}
  \item User Model: the conceptual model that the user develops in mind to
    explain and infer about the behavior of the device.
  \item Designer Model: the conceptual model that the designer develops to
    create the device.
  \item System Image: The user and designer communicate only through the
    system, so the system image is the catalyst that helps the user to develop
    the conceptual model in their mind. The system image must accurately
    represent the designer model, so that the user model can be as equivalent
    to the designer model as possible.
  \end{enumerate}
\item Manuals should be written first.
\end{enumerate}

\section*{The Principles of Good Design}
\begin{enumerate}
\item Visibility: By looking, the user can tell the state of the device and the
  alternatives for the action.
\item A good conceptual model: The designer provides a good conceptual model
  for the user, with consistency in the presentation of operations and results
  and a coherent, consistent system image.
\item Good mappings: It is possible to determine the relationships between
  actions and results, between the controls and their effects, and between the
  system state and what is visible.
\item Feedback: The user receives full and continuous feedback about the
  results of actions.
\end{enumerate}

\section*{Questions for Design Evaluation}
How easily can one:
\begin{enumerate}
\item Determine the function of the device?
\item Tell what actions are possible?
\item Determine mapping from intention to physical movement?
\item Perform the action?
\item Tell if the system is in the desired state?
\item Determine the mapping from system state to interpretation?
\item Tell what state the system is in?
\end{enumerate}
\end{document}

%%%%%%%%%%%%%%%%%%%%%%%%%%%%%%%%%%%%%%%%%%%%%%%%%%%%%%%%%%%%%%%%%%%%%%%%%%%%%%%
