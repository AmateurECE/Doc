%%%%%%%%%%%%%%%%%%%%%%%%%%%%%%%%%%%%%%%%%%%%%%%%%%%%%%%%%%%%%%%%%%%%%%%%%%%%%%%
% NAME:             Composite-Summary.tex
%
% AUTHOR:           Ethan D. Twardy <edtwardy@mtu.edu>
%
% DESCRIPTION:      This file 
%
% CREATED:          10/26/2018
%
% LAST EDITED:      06/15/2019
%%%

\documentclass{designdoc}

\lhead{A Summary of Composite Design}

\begin{document}
\section{Introduction}
This document serves to summarize the most important points from the book
{\it Composite/Structured Design}, by Glenford J. Meyers.

\section{The Nine Fundamental Problems}
Given below are nine of the most serious problems plaguing the software
industry. These problems are so omnipresent that they are titled the nine
{\it fundamental} problems.

\begin{enumerate}
\item The structure of the typical program is never designed. Rather, it is
  created in an ad hoc manner as it is written.
\item Most programs cannot adapt to changing requirements and environments.
\item Programmers spend the majority of their time correcting their mistakes.
\item Software reliability, once the subject of many jokes, is now a life and
  death matter.
\item The programming process is not improving as quickly as it could.
\item Technical decisions are rarely objective.
\item We do not stand on the shoulders of others.
\item A project of $n$ programmsers usually has $n$ objectives.
\item The program design phases are regarded too informally.
\end{enumerate}

\section{Terms Used in this Summary}
\begin{enumerate}
\item Function: what a module does; usually stated in verb/object form. (e.g.
  ``Carve the roast beast'').
\item Logic: How a module performs its function.
\item Context: Module use case and conditions
\item Subordinate Module: $A$ is subordinate to $B$ if $B$ calls $A$.
\item Superordinate Module: $A$ is superordinate to $B$ if $A$ calls $B$.
\item Fan-in: The number of distinct modules that call a module.
\item Fan-out: the number of distinct modules called by this module.
\item Homogeneous Datum: A scalar, a list with single field per entry, or an
  array where each element has the same meaning.
\item Restricted Module: A module that is over-specialized.
\end{enumerate}

\section{HIPO Diagrams}
A particularly useful tool for the software engineer is the HIPO
({\bf H}ierarchy {\bf I}nput {\bf P}rocess {\bf O}utput) diagram. As it is so
named, this diagram illustrates the hierarchy of calling modules, and is
usually accompanied by a table which describes inputs and outputs of each
module.

\section{A Good Design}
A good design usually has the three following characteristics:
\begin{enumerate}
\item The program is partitioned into parts having identifiable and
  understandable boundaries.
\item The design is hierarchical in organization.
\item The design maximizes dependence within a module and minimizes the
  dependence between modules.
\end{enumerate}

In addition, the hierarchy should usually be balanced (i.e. not too stretched
in one particular dimension). This is not always the case, but a balanced
hierarchy is a good code smell. Finally, modules should be maximally
independent. The highest form of coupling is no direct coupling.

Design processes, however, are just as important. A good design process leads
an engineer to design a strong system without thinking explicitly about the
strength of the design.

\section{Problems Encountered While Impelementing Modularity}
\begin{enumerate}
\item A module performs too many functions, which causes its logic to be
  obscured.
\item Common functions are distributed throughout many modules instead of being
  isolated in one place.
\item Modules interact on global state in unexpected ways.
\end{enumerate}

\section{Module Strength}
A way to quantify (roughly) the strength of relationships within a module.
There are seven categories of module strength, which are listed below in order
from strongest to weakest. It's important to note that sometimes, it is
acceptable or even necessary to allow a module with weaker strength, to
minimize weakness in another part of the program. A module has the highest
type of strength whose definition it meets.
\begin{enumerate}
\item Functional Strength, Informational Strength.
\item Communicational Strength
\item Procedural Strength
\item Classical Strength
\item Logical Strength
\item Coincidental Strength
\end{enumerate}

\subsection{Coincidental Module}
A module with coincidental strength either has a function which cannot be
defined, or it performs multiple completely unrelated functions. The function
\texttt{main()} is not coincidental because it usually has the function of the
program itself.

\subsection{Logical Module}
A logical strength module performs a number of functions, one of which is
explicitly selected by the caller. Note that in modern terms, this is referred
to as {\it Command Pattern}.

\subsection{Classical Module}
A module with classical strength performs multiple sequential functions where
there is a weak, but nonzero, relationship among the functions. A common
example of a module with classical strength is an ``Initialization'' module,
which does such things as: open the transaction record, open the master record,
initialize the summary table, print the report headings. These functions are
obviously related, but not strongly enough to warrant a single module dedicated
to all of them.

\subsection{Procedural Module}
Performs multiple sequential functions, where the sequential relationship
among these functions is implied by the problem or application statement. For
example, if the problem statement implies that when an invalid transaction is
encountered, the program should skip to the next transaction file, checkpoint
the master file, and display an appropriate error message, and a single
module is written which performs all of these functions, that module has
procedural strength.

\subsection{Communicational Module}
A communicational strength module is a module that fits the description of a
procedural module, {\it and} there is a data relationship among all of the
functions.

\subsection{Functional Strength}
A module with functional strength performs a single, specific function. In
addition, it is a module that does not conform to any of the qualifications of
a lesser form of strength. If a set of functions can be collectively described
as a single specific function in a coherent way, the module performing these
functions has functional strength.

\subsection{Informational Strength}
This type of module is the standard object-oriented class, as long as it is
efficiently designed. An informational strength module has a single interface,
which encapsulates all of its data and related functions. Myers suggests that
software should be designed for functional strength modules, which should be
grouped into informational strength modules where appropiate.

\section{Module Coupling}
Is a notation for quantifying the dependency between modules. There are seven
types of coupling, listed below in order from best case to worst case.
\begin{enumerate}
\item No Direct Coupling
\item Data Coupling
\item Stamp Coupling
\item Control Coupling
\item External Coupling
\item Common Coupling
\item Content Coupling
\end{enumerate}
As with module strength, we will cover these from the bottom up.

\subsection{Content Coupling}
Where one module directly references the internals of another module. An
example is the relationship between coroutines. Spier has a fascinating
story about this, in {\it Software Malpractice--A Distasteful Experience}
(1976).

\subsection{Common Coupling}
When two or more modules reference the same global data structure.

\subsection{External Coupling}
A group of modules have external coupling when they reference a datum with
\texttt{extern} linkage.

\subsection{Control Coupling}
One module explicitly passes elements of control to another module. ``Elements
of control'' are arguments which control the behavior or a logical strength
module. This is, as noted in the subsection on logical strength, commonly
referred to as {\it Command Pattern}.

\subsection{Stamp Coupling}
Two or more modules have stamp coupling if they reference the same nonglobal
data structures.

\subsection{Data Coupling}
Two ore more modules have data coupling if they do not have a lesser form of
coupling, the modules directly communicate with each other (i.e. one calls
another), and all interface data are homogeneous data items.

\section{How to Minimize Stamp Coupling}
One way to remove stamp coupling is to replace it with data coupling, where
possible. If a module only needs access to one or a few fields of a struct,
it is better to pass only those fields than to pass the whole struct. Another
way is to use informational strength modules. If all transformations on a
structure can be isolated to a single informational strength module, then all
stamp coupling via this structure is eliminated. Finally, it is possible to
reduce stamp coupling by passing pointers to structs. In this way, modules that
must only pass on the information without modifying it are not privy to the
structure of the info, and therefore not unnecessarily stamp coupled through a
superordinate and subordinate module.

\subsection{An Example: Coupling Tradeoffs}
Many programs not developed in languages that support exception stacks
experience difficulty in architecting an error reporting infrastructure. There
are two common strategies in use. The first is to place the text of each
message in the modules that detect the error condition. Each module will write
the message directly (in some fashion). This is stamp coupling--the error
writing mechanism references the data that are set by the calling module.
However, this mechanism does not guarantee against redundancy or consistency.
The second strategy is to place texts of all messages in a single module. Each
module that detects an error will call this module and pass information to
select the error. This is control coupling, which is a lower form of
coupling--however it is usually recommended, because it better promotes
information hiding and guards against redundancy.

\section{Some Miscellaneous Tips}
\begin{enumerate}
\item Module Size: This should not be a primary target, but it does help with
  code legibility. Generally, files should be less than a thousand lines, and
  each particular module should be less than 20 statements.
\item Syntax parsing and semantics interpretation are two different functions:
  \begin{enumerate}
  \item Only one module should be sensitive to the overall syntactic structure
    of the commands so that a change to this structure requires changing only
    one module.
  \item Only one module should know the command names so that adding new
    commands requires changing only one existing module.
  \item Only one module should be sensitive to the keywords and operands of
    each particular command or command type.
  \end{enumerate}
\item Be wary of situations where an interface does not completely match the
  function of its module. This is an indication of low strength.
\item Strive for high average module fan-in. (i.e. no repetition or spreading
  logic into other modules).
\item Design general modules which can be reused.
\item Modules shall have no internal state which will remain between function
  calls. Modules of this type are called {\it Predictable}.
\item Try for reentrant and recursive-safe modules wherever possible.
\item Don't over-constrain, over-specialize, or over-generalize modules.
  Over-specialized modules are referred to as {\it Restricted} modules.
\end{enumerate}

\section{The Design Thought Process}
In this section, we will cover module decomposition, or the act of decomposing
a module into subordinate modules recursively. There are three major types of
decomposition, with a quaternary method also mentioned. The particular
technique is chosen based on the type of the problem. Composite design is a
recursive divide-and-conquer problem. These decomposition techniques can
therefore be applied recursively until the design is complete. When the logic
of a module can easily be visualized, it probably needs no more than one level
of further decomposition--and at that, probably no further decomposition. It's
important to note that the designer should define the interfaces between the
modules before performing another iteration of composition.

Composite design is a process of stepwise refinement. If decomposing a
particular module provides some further insight into how superordinate modules
should be designed, the designer should not ignore this advice. It's common and
not unreasonable to design a problem twice. The best way to become familiar
with a problem is to design a solution for it.

\subsection{Source/Transform/Sink Decomposition}
This decomposition strategy should be used first. This scheme has four steps,
outlined below. The idea behind STS Decomposition is to imagine a module as a
logical process that refines an input data stream into an output data stream;
the module takes a number of inputs, performs some logical process, then
disperses the output data to the calling module.
\begin{enumerate}
\item Outline the structure of the problem:
  Identify three to ten subproblems. Arrange these in the order that represents
  the flow of data. If this is impossible, consider using another
  decomposition.
\item Define immediate submodules:
  \begin{enumerate}
  \item Identify the streams of input data and output data:
  \item Identify the point where the input stream stops existing as a logical
    entity (i.e. acquire the input stream in its most abstract form).
  \item Transform the major input stream in its most abstract form into the
    major output stream in its most abstract form.
  \item Disperse the major output stream.
  \end{enumerate}
\end{enumerate}

\subsection{Transactional Decomposition}
This decomposition scheme is used for data-driven modules. It works by
decomposing a function into a set of immediate subordinate modules based on
the syntax of the input data. As an example, consider the function ``Apply
a transaction to the record file.'' This module is a prime candidate for
transactional decomposition because the behavior of this module depends on the
type of transaction that is applied. Notice that since the caller does not
explicitly set the behavior of the module, this module does not have control
coupling. If the module which applies the transactions to the record file is
named {\it Applicator}, its subordinate modules may be called
{\it Sale Applicator}, {\it Return Applicator}, {\it Order Applicator}, etc.

\subsection{Functional Decomposition}
This is the simplest form of decomposition. It works by pulling single
subfunctions from a module to achieve certain purposes. These purposes are
usually:
\begin{enumerate}
\item Isolating common subfunctions.
\item Isolating functions within an informational-strength module.
\end{enumerate}

\subsection{Data Structure Decomposition}
This type of decomposition depends on {\it data structure}, whereas STS
Decomposition depends on {\it data flow}. It is not described in particular
depth in the book, because the languages in common use at the time of its
publishing did not support Object Oriented programming in the modern sense. The
steps involved in data structure decomposition are described below.
\begin{enumerate}
\item Define the substructures of the input and output data.
\item Find one-to-one relationships between components of those data
  structures. If this cannot be done, define intermediate data structures.
\item Form a structure based on these relationships.
\item Allocate elementary operations to appropriate components.
\end{enumerate}
In general, this technique restricts one's thinking to four components: elementary, sequence, iteration, selection. An {\it elementary} component is an atomic
datum; for example, a primitive type. A {\it sequence} component has more than
one part, each appearing only once and in order; for example, a
\texttt{struct}. An {\it iteration} component has a single part that occurs
zero or more times; for example, an array. A {\it selection} component has more
than one part, only one of which is present in each occurrence of the selection
component; for example, a \texttt{union}. This decomposition technique is
described further in {\it Principles of Program Design}, by M.A. Jackson
(1975).

\section{Post Design Review}
Any design should be reviewed after it is believed to be finished. A number of
questions and thought experiments are provided below to aid in this review
process.

\subsection{Structure Optimization}
In this process, the general structure of the program is reviewed. As an
example, one should ask the question ``Has module independence been
maximized?'' To answer this question, one may go through the program and search
for implicit or explicit dependencies, then consider eliminating these using an
informational strength module. As another point of review, two more points of
trade off are considered. Firstly, perhaps one should consider decomposing
further before wrapping unrelated functions in an informational module.
Secondly, one should prefer to exclude related functions rather than introduce
a lesser form of coupling.

\subsection{Static Code Review}
In the context of this book, {\it static review} is a process by which the
design of a program is visually inspected to check quality and correctness.
This type of review tends to entail an independent observer asking questions of
the designer. It is important to note that the purpose of the review is to find
weaknesses in the design, not to make hot patches or on-the-fly improvements.
Corrections of weakness should be deferred until after the review. Below is a
reasonable set of questions to pose at a static review:

\begin{enumerate}
\item Does each module have functional or informational strength? If any do
  not, why not? The designer should be expected to objectively justify any
  tradeoffs which were made that result in a lesser form of strength.
\item Is each pair of modules either data coupled or not directly coupled? If
  any are not, why not?
\item Is the decomposition complete? That is, can the logic of each module be
  easily visualized?
\item Where a module or entry point has a fan-in greater than one, are the
  interface definitions to the module consistent? That is, does each interface
  to the module have the same number of input arguments and the same number of
  output arguments? Does each corresponding argument in each interface have the
  same attributes?
\item Are there any redundant data in any of the module interfaces? Redundant
  data contributes to confusion about the module's use; however it may be
  useful or even necessary in some cases.
\item Are the interface data to each module consistent with the definition of
  the module's function?
\item Are there multiple modules in the program that seem to perform the same
  function?
\item Is each module described by its function rather than by its context or
  logic? Has each function been stated correctly?
\item Are there any preexisting modules that could have been used in this
  program?
\item Are there any restrictive modules (i.e., modules that are unnecessarily
  overspecialized)?
\item Are all moduled predictable (without state)?
\item Are there instances where recursion either should have been used or
  should not have been used?
\item Is there any aspect of the design that is precluded by the programming
  language to be used?
\item Is the hierarchical structure too extreme (i.e. is the tree particularly
  unbalanced in any way)? This is not directly an issue, but is a code smell
  that should be explored.
\item Has anything been left out? Are all functions in the external
  specification reflected in the program?
\item Has anything been included that should not be? Does the program do more
  than that stated in the external specification?
\item Is the design obscure? Does it contain anything that might be easily
  misunderstood?
\item Have any unstated assumptions been made?
\end{enumerate}

\subsection{Dynamic Design Review}
In this section, a few use cases are outlined and the program is ``mentally''
stepped through (using a whiteboard) to verify that the design covers certain
edge cases and implements business logic correctly.

\section{After Optimization \& Verification}
After design review has been conducted, the designer should perform module
interface design of all modules. This includes expressing all of the
information needed to invoke the module, and nothing else (i.e., write the
prototypes of each function and constructor in each module).

\section{Final Notes on Composite Design}
As a few final notes, designers should put more stock in qualitative feedback
on design processes than in quantitative feedback. This is mostly because the
standards of the qualitative feedback are unclear or not given. In the past,
NASA has used Composite Design in the design of compilers, operating systems,
and applications for the space shuttle.

\subsection{Components of Composite Design}
\begin{enumerate}
\item A set of concepts about program structure.
\item A set of design guidelines, measurements, and goals.
\item A set of thought processes.
\item Notation for describing program structure.
\end{enumerate}

\subsection{Reasons to Use Composite Design}
\begin{enumerate}
\item In general, programs created using composite design are easier to change
  and maintain.
\item Composite Design promotes increased reliability of programs.
\item Composite Design may result in lower development costs.
\item Composite Design promotes module reuse in future programs.
\end{enumerate}

\subsection{Objectives of Program Maintenance}
Whenever a change to the program is performed, the following objective should
be kept:
\begin{enumerate}
\item Have a global view of the program and a long-range view of the
  consequences of each change to lessen the chances of deteriorating the
  program's structure.
\item Modify the program in such a way that the resultant program has the
  appearance of being originally designed and coded that way.
\end{enumerate}

\section{Further Reading}
\begin{enumerate}
\item G.J. Meyers, ``Software Reliability: Principles and Practices''.
\item M.J. Spier, ``Software Malpractice--A Distasteful Experience'', 1976.
\item M.A. Jackson, ``Principles of Program Design'', 1975.  
\end{enumerate}

\end{document}

%%%%%%%%%%%%%%%%%%%%%%%%%%%%%%%%%%%%%%%%%%%%%%%%%%%%%%%%%%%%%%%%%%%%%%%%%%%%%%
