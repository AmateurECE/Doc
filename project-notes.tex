%%%%%%%%%%%%%%%%%%%%%%%%%%%%%%%%%%%%%%%%%%%%%%%%%%%%%%%%%%%%%%%%%%%%%%%%%%%%%%%
% NAME:             project-notes.tex
%
% AUTHOR:           Ethan D. Twardy <edtwardy@mtu.edu>
%
% DESCRIPTION:      A dumb list of notes for ongoing projects.
%
% CREATED:          05/01/2019
%
% LAST EDITED:      05/14/2019
%%%

\documentclass[format.tex]{subfiles}

\begin{document}
\section*{``Short'' Projects}
\subsection*{Resume in \LaTeX}
This is simple enough. Rewrite the resume in \LaTeX.

\subsection*{SSH Keys}
I used to have SSH keys set up to allow me to easily startup an SSH session on
my Linux server at home. It should be easy enough to set that up again.

\subsection*{Fontify Minted Blocks in \LaTeX}
Set up an Emacs \LaTeX \texttt{polymode} to fontify blocks of text in
\texttt{minted} blocks in \LaTeX source files. Should support basic languages
like Bash, C/C++, Python, and R.

\subsection*{Update Repository}
The Repository is a cache of information that may be useful to me in the
future. Currently, the Repository is in a transitionary period. Ideally, it is
published using GitHub static web pages, and is simply a linked collection of
PDF and plain text files. Also, all the PDFs will be generated from \LaTeX
sources that all use a common class file.

\subsection*{Backup The Passport}
This needs to be done regularly. It's time to do it again.

\subsection*{New House Server Setup Checklist}
This is a document that goes in the Repository. It's purpose is to provide a
step-by-step guide to setting up the server in a new house.

\subsection*{Podscraper}
A simple web-scraping script in Python to pull down some of my favorite
podcasts.

\subsection*{LaunchServices}
This is a simple Swift application to change which applications are used to
open certain files. I want to use this to make \texttt{/usr/bin/open} open
PDF files using Chrome instead of Preview.

\subsection*{SecurityTransforms}
The idea is to create a robust and cross-platform encryption tool that will be
long-living. This project was motivated because Apple recently deprecated
OpenSSL in favor of its own encryption libraries. Unfortunately, these
libraries are not portable to Linux, and currently I have a solution worked out
using OpenSSL in a Docker container, but that won't last forever.

\subsection*{VNC On the Server}
Set up a VNC Server on the Rock64, so that I can run hefty software across the
WAN, such as Ghidra, or the Lattice Diamond Programmer.

\subsection*{Rewrite sortmusic}
This program is a mess, in may senses of the word. Rewrite it to be more
maintainable, user-friendly, etc.

\subsection*{Minicom Patch}
While working on
\href{https://github.com/AmateurECE/SerialBridge}{SerialBridge}, I noticed that
minicom doesn't actually check to ensure that the baud rate was set correctly.
I stumbled upon this while trying to set the baud rate of the serial port to
1.5Mbaud. Minicom reported that the interface supposedly \textit{was} operating
at 1.5Mbaud, however I was able to empirically confirm that it was in fact
\textit{not}. This turned out to be because the driver did not support
1.5Mbaud. By digging through the source, I determined that the check should go
at the call to \mintinline{C}{ioctl} at
\href{https://github.com/Distrotech/minicom/blob/54202fe0ea8510dc8fcd23ab49d39%
  d3d8cb2e529/src/sysdep1.c#L519}{line 519}:
\begin{minted}{C}
  ioctl(fd, TIOCSETP, &tty);
\end{minted}

\subsection*{Update Current Projects}
Have you checked the output of
\href{https://github.com/AmateurECE/Sysgit}{Sysgit} lately? Run the script to
locate repositories on the system that need updating.

\subsection*{Check forcefsck}
% TODO: Check forcefsck

\subsection*{Fix dma-names in dtb}
% TODO: Fix dma-names in dtb

% TODO: ServerSearcher
% - A launchctl service that sends a packet to the broadcast IP whenever a new
% - network is joined. The server would have a daemon listening on a particular
% - port for this particular message. The server would then respond to the
% - source IP with its own IP address. The launchctl service would then e.g.
% - put this IP address into a file in /var/ or something.

\end{document}

%%%%%%%%%%%%%%%%%%%%%%%%%%%%%%%%%%%%%%%%%%%%%%%%%%%%%%%%%%%%%%%%%%%%%%%%%%%%%%
