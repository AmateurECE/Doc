%%%%%%%%%%%%%%%%%%%%%%%%%%%%%%%%%%%%%%%%%%%%%%%%%%%%%%%%%%%%%%%%%%%%%%%%%%%%%%%%
% NAME:             numerical-problem.tex
%
% AUTHOR:           Ethan D. Twardy
%
% DESCRIPTION:      This file describes the problem of solving a system of
%                   unknown equations, numerically.
%
% CREATED:          11/14/2017
%
% LAST EDITED:      11/15/2017
%%%

\documentclass[11pt]{article}

\begin{document}
Heyo. I've been thinking about this for a couple of days. I think there must be
a method for something like this, but I lack the background for it.

Given a system of linear equations

% Inputs:
% A = [ f,  Vout,   Vf, Vin,    Vsat,   i_out,  v_ripple]
% A = [ a_1, a_2, a_3, a_4, a_5, a_6, a_7 ]

% c_t = f(f, Vout, Vf, Vin, Vsat)
% r_sc = f(i_out)
% l_min = f(Vin, Vsat, Vout, i_out, f, Vf)
% c_o = f(i_out, f, v_ripple)

% f(a_1, a_2, a_3, a_4, a_5)
% f(a_6)
% f(a_1, a_2, a_3, a_4, a_5, a_6)
% f(a_1, a_6, a_7)

% Outputs:
% B = [ c_t,    r_sc,   l_min,  c_o ]
% B = [ b_1, b_2, b_3, b_4 ]
\end{document}

%%%%%%%%%%%%%%%%%%%%%%%%%%%%%%%%%%%%%%%%%%%%%%%%%%%%%%%%%%%%%%%%%%%%%%%%%%%%%%%%
